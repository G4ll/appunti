\chapter{Il problema di Goldbach}
Approcciamo ora le congetture di Goldbach, dimostrandone una forma debole.

\section{Richiami sui numeri primi}
Ci serviranno alcuni risultati classici, stile PNT.

\begin{definition}
    La funzione $\Lambda$ di von Mangoldt è data da $$\Lambda(n)=\begin{cases}
    \log p & \text{se } n=p^a \\
    0 & \text{altrimenti}
    \end{cases}$$
    Sia poi $\psi(x)=\sum_{n\le x}\Lambda(n)$ la funzione $\psi$ di Chebycheff.\\
    Infine definiamo la $\theta$ di Chebycheff: $\theta(x)=\sum_{p\le x}\log p$ dove la somma è fatta sui primi.
\end{definition}
\begin{proposition}
    Si verificano facilmente le seguenti proprietà:
    \begin{itemize}
        \item $e^{\psi(N)}=mcm(1,2,\dots,N)$.
        \item $\psi(x)=\sum_{i=1}^{\lfloor\log_2(x)\rfloor}\theta(x^{1/i})$.
    \end{itemize}
\end{proposition}

Un teorema classico è il seguente
\begin{theorem}
    $x \ll \psi(x), \theta(x) \ll x$.
\end{theorem}
che raffinato diventa il teorema dei numeri primi
\begin{theorem}[PNT]\label{pnt}
    $\psi(x)\sim \theta(x) \sim x$.
\end{theorem}

\section{Il teorema di Vinogradov}
L'oggetto di studio di questa sezione è la seguente funzione
\begin{definition}
    Dato $N\ge2$, sia $\ds r(N)=\sum_{k_1+k_2+k_3=N}\Lambda(k_1)\cdot\Lambda(k_2)\cdot\Lambda(k_3)$
\end{definition}
L'obiettivo finale sarà dimostrare il
\begin{theorem}[Vinogradov, 1930]Per ogni $A>0$ vale
    $$ r(N)=\frac12\sigma(N)\cdot N^2+O\left( \frac{N^2}{(\log N)^A} \right) $$
    dove $\ds\sigma(N)= \prod_{p\mid N}\left( 1-\frac{1}{(p-1)^2} \right) \cdot \prod_{p\nmid N}\left( 1+\frac{1}{(p-1)^3} \right)$
\end{theorem}
\begin{oss}
    Se $N$ è pari, allora $\sigma(N)=0$, quindi la parte principale svanisce e occorre studiare meglio il resto.
\end{oss}

Vediamo intanto come il teorema risolve la congettura di Goldbach sui dispari.

Detta $r^\ast(N)=\sum_{p_1+p_2+p_3=N}\log(p_1)\log(p_2)\log(p_3)$, si può vedere che è molto vicina lla $r(N)$ che stiamo studiando.\\
Infatti, consideriamo i termini di $r(N)$ in cui almeno un $k_i$ (diciamo $k_1$) è una potenza di un primo con esponente almeno $2$; ma allora deve essere $k_1\le\sqrt{N}$ e quindi si ricava $r(N)-r^\ast(N)\le \sum_{k_1\le\sqrt{N}}\Lambda(k_1)\cdot \sum_{k_2+k_3\le N}\Lambda(k_2)\Lambda(k_3)$.\\
Il primo fattore è esattamente $\psi(\sqrt{N})$, che per il PNT è $\ll\sqrt N$; il secondo fattore può essere maggiorato con $\sum_{k_2\le N}\Lambda(k_2)\cdot\log N$, ovvero $\psi(N)\log N$ che di nuovo è $\ll N\log N$.\\
Concludiamo cioè che $r(N)=r^\ast(N)+O(N^{3/2}\log N)$.\\
Abbiamo allora il
\begin{corollary}
    Ogni intero dispari $N$ sufficientemente grande è somma di $3$ primi in almeno $c\dfrac{N^2}{\log^3N}$ modi, con $c>0$.
\end{corollary}
\begin{proof}
    Per il teorema di Vinogradov e la stima appena vista, la parte principale di $r^\ast(N)$ è $c N^2$.\\
    Inoltre $\ds r^\ast(N)=\sum_{p_1+p_2+p_3=N}\log(p_1)\log(p_2)\log(p_3) \le \sum_{p_1+p_2+p_3=N}\log^3(N)$, cioè il numero di modi di scrivere $N$ come somma di $3$ primi è almeno $\frac{r^\ast(N)}{\log^3(N)}$.
\end{proof}

Per la dimostrazione del teorema di Vinogradov ci serviranno un po' di lemmi.

L'idea comunque è di considerare la funzione $S(\alpha,N)=\sum_{k\le N}\Lambda(k)e(\alpha k)$, da cui $S^3(\alpha,N)=\sum_{l\le 3N}e(\alpha l) r(l,N)$.\\
Quindi per inversione di Fourier possiamo scrivere
$$ r(N)=\int_{0}^{1}S^3(\alpha,N)e(-\alpha N)d\alpha $$

Dividiamo in archi principali e secondari...



