\chapter{Introduzione}
In questo corso tratteremo prevalentemente la teoria analitica additiva; in particolare useremo il metodo del cerchio di Hardy-Littlewood per approcciarci ai seguenti problemi:
\begin{itemize}
    \item problema di Waring
    \item problema di Goldbach
    \item problema di Erdos, Roth, Szemeredi
\end{itemize}
\section{Il problema di Erdos}
L'ultimo problema è stato proposto da Erdos nella seguente forma:
\begin{congettura}[Erdos]
    Sia $E\subset\N$ un insieme tale che $\overline d(E)=\limsup_{N\to\infty}\frac{\# E\cap[1,N]}{N}>0$. Allora esistono tre elementi di $E$ in progressione aritmetica.
\end{congettura}
Il problema in questa forma venne risolto da Roth nel 1953; in seguito venne mostrato il seguente
\begin{theorem}[Szemeredi, 1975]
    Sia $E\subset\N$ un insieme tale che $\overline d(E)>0$. Allora esistono segmenti di progressioni aritmetiche arbitrariamente lunghi.
\end{theorem}
Negli ultimi anni si è arrivati anche al seguente risultato
\begin{theorem}[Green e Tao, 2004]
    L'insieme dei numeri primi contiene progressioni aritmetiche arbitrariamente lunghe
\end{theorem}

\section{Il problema di Waring}
Un risultato molto importante nella teoria elementare dei numeri è il famoso ``teorema dei quattro quadrati'', ovvero
\begin{theorem}[Lagrange, 1770]
    Ogni intero positivo si può scrivere come somma di al più quattro quadrati.
\end{theorem}
Nello stesso anno, Waring propose la seguente generalizzazione:
\begin{congettura}[Waring, 1770]
    Ogni intero positivo si scrive come somma di al più 9 cubi, 19 potenze quarte, e così via...
\end{congettura}
Questa frase ci porta a definire il nostro oggetto di studio:
\begin{definition}
    Fissato $k$, sia $g(k)$ il minimo intero (eventualmente infinito) tale che ogni $n\in\N$ si può scrivere come somma di $g(k)$ potenze $k$-esime.
\end{definition}
Uno dei primi risultati su $g(k)$ è un bound dal basso:
\begin{proposition}[Eulero]
    $\ds g(k)\ge2^k+\left\lfloor \left(\frac{3}{2}\right)^k \right\rfloor -2$
\end{proposition}
\begin{proof}
    Sia $n_k=2^k\cdot\left\lfloor \left(\frac{3}{2}\right)^k \right\rfloor-1$; si vede facilmente che $n_k<3^k$. Quindi per scriverlo come somma di potenze $k$-esime possiamo usare solamente $1^k,2^k$.\\
    Tuttavia $2^k\cdot\left\lfloor \left(\frac{3}{2}\right)^k \right\rfloor>n_k$, perciò possiamo usare al più $\left\lfloor \left(\frac{3}{2}\right)^k \right\rfloor-1$ volte il $2^k$.\\
    Rimane poi $n_k-2^k\cdot\left( \left\lfloor \left(\frac{3}{2}\right)^k \right\rfloor -1 \right)=2^k-1$, per cui possiamo usare solo gli $1^k$, e ce ne servono $2^k-1$.\\
    Sommando le due quantità otteniamo ul bound cercato.
\end{proof}
\begin{oss}
    Sebbene questo sembri un bound banale, in realtà è molto forte: se calcoliamo il valore del bound per $2,3,4$ otteniamo $4,9,19$ che sono esattamente i valori congetturati da Waring.
\end{oss}
Nell'ultimo secolo si è infatti dimostrato che
\begin{theorem}[Mahler, 1957]
    $\ds g(k)=2^k+\left\lfloor \left(\frac{3}{2}\right)^k \right\rfloor -2$, tranne al più un numero finito di $k$.
\end{theorem}
Un importante risultato di inizio secolo è il seguente
\begin{theorem}[Hilbert, 1909]
    Il numero $g(k)$ esiste finito per ogni intero $k$.
\end{theorem}

Dato che lo studio di $g(k)$ è quasi completamente risolto, si è iniziata a studiare un'altra quantità
\begin{definition}
    Si indica con $G(k)$ il minimo intero $s$ tale che ogni intero sufficientemente grande è scrivibile come somma di $s$ potenze $k$-esime.
\end{definition}
\begin{oss}
    Vale ovviamente $G(k)\le g(k)$ e quindi anche $G(k)$ è finito $\forall k$.
\end{oss}
Lo studio di questa funzione è molto più difficile di quello di $g(k)$. Alcuni dei risultati che si hanno sono
\begin{theorem}[Davenport, 1939]
    $G(4)=16$
\end{theorem}
\begin{theorem}[Vaughan e Wooley]
    $G(k)\le k\log k+k\log\log k+ Ck$
\end{theorem}


\section{Il problema di Goldbach}
Questo è uno dei problemi più famosi della matematica, data la semplicità dell'enunciato:
\begin{congettura}[Goldbach, 1742]$ $
    \begin{itemize}
        \item Forma forte: Ogni intero pari è esprimibile come somma di due primi.
        \item Forma debole: Ogni intero è scrivibile come somma di al più tre primi.
    \end{itemize}
\end{congettura}

Un risultato parziale è il seguente
\begin{theorem}[Helfgott, 2013]
    Ogni intero dispari $n\ge7$ si scrive come somma di tre primi dispari.
\end{theorem}

Ci sono metodi ``probabilistici'' per vedere che asintoticamente molti numeri soddisfano la congettura.\\
Ad esempio, Hardy e Littlewood dimostrarono che il numero di rappresentazioni come somma di $k$ primi è asintotico a $c_k \frac{n^2}{\log^3n}$; tuttavia $c_2=0$, quindi la parte principale è un'altra.
Abbiamo poi il seguente
\begin{theorem}
    Ogni intero positivo, tranne al più un insieme $E$, è somma di $2$ primi; con $E\cap[1,N]=O\left( \frac{N}{\log^\alpha N} \right)$ per ogni $\alpha$.
\end{theorem}

Un'altra strada è attraverso metodi di crivello, giungendo a risultati del tipo
\begin{theorem}[Chen, 1973]
    Ogni intero positivo sufficientemente grande è somma di un primo e di un semiprimo (ovvero di un prodotto di al più due primi).
\end{theorem}


\section{Il metodo di Hardy-Littlewood}
Sia $a_m$ una successione crescente di interi; siamo interessati a studiare il comportamento della quantità $R_s(n)$ che è il numero di rappresentazioni di $n$ come somma di $s$ termini della successione.

Introduciamo allora la serie di potenze $F(z)=\sum_{m\ge0} z^{a_m}$. Vale allora
$$ F^s(z)=\sum_{n\ge0}z^n\cdot R_s(n)$$

Dato che $F$ è olomorfa in $|z|<1$, possiamo usare il teorema di Cauchy per ottenere
$$ R_s(n)=\oint_{|z|=\rho}\frac{F^s(z)}{z^{n+1}}dz$$

\begin{notazione}
    Definiamo $e(\alpha)=e^{2\pi i\alpha}$. Diciamo che $f\ll g$ se $f=O(g)$.
\end{notazione}
